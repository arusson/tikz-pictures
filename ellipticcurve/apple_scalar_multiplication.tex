%
% I made this graph to explain how Apple's implementation
% of the Montgomery ladder scalar multiplication deals
% when the first most significant bits of the scalar are zeros.
% The whole explanation is in section 3.2.2 of my thesis (in French),
% available here: https://tel.archives-ouvertes.fr/tel-03663532/
% (the algorithm and graph are on page 31).
%
% Also, Apple finally released in 2022 the source code of this version
% of Montgomery ladder with explanation in commits (in English),
% which is available to download here: https://developer.apple.com/security/
% (link to corecrypto source at the bottom).
%

\documentclass[tikz]{standalone}
\usepackage[utf8]{inputenc}

\definecolor{DarkRed}{rgb}{0.7,0.2,0.2}

\begin{document}

\tikzstyle{noeudcoz}=[rectangle,fill=DarkRed!10,minimum width=3cm, rounded corners=8pt,draw]

\begin{tikzpicture}[xscale=1.4,yscale=2.1]\sffamily
\node[draw,rectangle,rounded corners=2pt] (start) at (0,4) {start};
\node[noeudcoz] (0) at (-2.5,3) {($2$, $1$), ($1$, $0$, $0$)};
\node[noeudcoz] (1) at (2.5,3) {($2$, $1$), ($0$, $1$, $1$)};
\node[noeudcoz] (00) at (-1.5,2) {($-2$, $1$), ($0$, $0$, $0$)};
\node[noeudcoz] (000) at (-1.5,1) {($2$, $1$), ($0$, $0$, $0$)};
\node[noeudcoz] (001) at (1.5,1) {($2$, $-1$), ($1$, $1$, $1$)};
\node[noeudcoz] (01) at (-4,2) {($-2$, $1$), ($1$, $1$, $1$)};
\node[noeudcoz] (010) at (-4,0) {($-2$, $-3$), ($0$, $1$, $0$)};
\node[noeudcoz] (011) at (-1.5,0) {($-4$, $-3$), ($0$, $1$, $1$)};
\node[noeudcoz] (10) at (1.5,0) {($2$, $3$), ($0$, $1$, $0$)};
\node[noeudcoz] (11) at (4,0) {($4$, $3$), ($0$, $1$, $1$)};

\draw[-latex]
(start) edge node[above] {0} (0)
        edge node[above] {1} (1)
(0) edge node[right] {0} (00)
    edge node[left] {1} (01)
(00) edge[bend left] node[right] {0} (000)
     edge node[above] {1} (001)
(000) edge[bend left] node[left] {0} (00)
      edge node[above] {1} (01)
(001) edge node[left] {0} (10)
      edge node[above] {1} (11)
(01) edge node[left] {0} (010)
     edge[bend right=20] node[right] {1} (011)
(1) edge[bend left=48] node[left] {0} (10)
    edge[bend left] node[right] {1} (11);
\draw[dotted] (010) --++ (0,-.5);
\draw[dotted] (011) --++ (0,-.5);
\draw[dotted] (10) --++ (0,-.5);
\draw[dotted] (11) --++ (0,-.5);
\end{tikzpicture}

\end{document}